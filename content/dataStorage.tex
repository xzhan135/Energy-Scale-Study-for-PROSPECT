\subsection{Antineutrino Detector Data}

Raw data are stored using LLNL's HPSS service.
Within about an hour of data being taken, a copy is made on the HPSS and from that point on serves as the archival copy.
The original copy of the raw data is removed as needed to reclaim storage space on the DAQ computers.
This removal is done with a custom software tool that checks that the HPSS copy exists and has an identical checksum to the original copy,
	and only deletes the original if the HPSS copy is confirmed to be identical.
Once that occurs, the HPSS copy is the only copy of the raw data.
\rvnote{Are the Unpacked files stored on the LLNL HPSS tapes?}

A secondary archive exists at Yale.
Instead of archiving the raw data, the Yale archive contains the Unpacked data.
Unpacked data losslessly contains all the information of the raw data, only differing in storage format.
As a result, the Yale archive can be used to recover the original raw data if needed in the event of a data loss at the LLNL HPSS archive.
\rvnote{How do the Unpacked files make their way to Yale?}

Processed data is kept on spinning disks at LLNL.
These disks are non-permanent storage subject to a deletion policy, reflecting our policy that processed data is not intended to be permanently preserved in all cases.
Instead, the record of the data processing allows processed data to be regenerated at any time using identical settings (including software version) as the original.

\rvnote{Does the data management plan require making any of the processed data available after a publication?  Will a copy of the processed data be tagged and saved when major publications are submitted?  How does the data storage interact with the blinding scheme?}


\subsection{Databases}

Run and calibration databases are stored as sqlite3 files.
The run database is periodically copied from the DAQ coordinator node to Yale, along with updated files.
The calibration database is generated fresh as part of the analysis process for each ``data release'' calibration pass,
	and kept with the associated calibrated data files.
\textbf{how are DB tables stored, backup, distributed, etc?}

\subsection{Code Repositories}

\textbf{TL to write a paragraph about the git repo?}
